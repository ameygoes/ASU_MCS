\documentclass{article}

\usepackage{fancyhdr, extramarks, amsmath, amsthm, amsfonts, tikz, palatino, enumerate, mdframed, xspace}

%
% Basic Document Settings
%
\topmargin=-0.45in
\evensidemargin=0in
\oddsidemargin=0in
\textwidth=6.5in
\textheight=9.0in
\headsep=0.25in

\linespread{1}

\pagestyle{fancy}
\chead{\hmwkClass\: \hmwkTitle}
\rhead{\firstxmark}
\lfoot{\lastxmark}
\cfoot{\thepage}

\renewcommand\headrulewidth{0.4pt}
\renewcommand\footrulewidth{0.4pt}

\setlength{\parindent}{0pt}
\setlength{\parskip}{1em}

\newcommand{\solution}[1]{
    \hfill
    \vspace{1mm}
  \begin{mdframed}
    \textbf{Solution:}
      #1
   \end{mdframed}
 }

% \newcommand{\solution}[1]{}

%
% Create Problem Sections
%
\newcommand{\enterProblemHeader}[1]{
    \nobreak\extramarks{}{Problem \arabic{#1} continued on next page\ldots}\nobreak{}
    \nobreak\extramarks{Problem \arabic{#1} (continued)}{Problem \arabic{#1} continued on next page\ldots}\nobreak{}
}

\newcommand{\exitProblemHeader}[1]{
    \nobreak\extramarks{Problem \arabic{#1} (continued)}{Problem \arabic{#1} continued on next page\ldots}\nobreak{}
    \stepcounter{#1}
    \nobreak\extramarks{Problem \arabic{#1}}{}\nobreak{}
}

\setcounter{secnumdepth}{0}
\newcounter{partCounter}
\newcounter{homeworkProblemCounter}
\setcounter{homeworkProblemCounter}{1}
\nobreak\extramarks{Problem \arabic{homeworkProblemCounter}}{}\nobreak{}

%
% Homework Problem Environment
%
% This environment takes an optional argument. When given, it will adjust the
% problem counter. This is useful for when the problems given for your
% assignment aren't sequential. See the last 3 problems of this template for an
% example.
%
\newenvironment{homeworkProblem}[1][-1]{
    \ifnum#1>0
        \setcounter{homeworkProblemCounter}{#1}
    \fi
    \section{Problem \arabic{homeworkProblemCounter}}
    \setcounter{partCounter}{1}
    \enterProblemHeader{homeworkProblemCounter}
}{
    \exitProblemHeader{homeworkProblemCounter}
}

\newcommand{\hmwkTitle}{Homework\ \#2 }
\newcommand{\hmwkClass}{CSE 575}

\title{
  \textmd{\textbf{\hmwkClass:\ \hmwkTitle}}\\
}


\renewcommand{\part}[1]{\textbf{\large Part \Alph{partCounter}}\stepcounter{partCounter}\\}

%
% Various Helper Commands
%

% For partial derivatives
\newcommand{\pderiv}[2]{\frac{\partial#1}{\partial #2}}

% Bold letters for vectors
\newcommand{\boldvec}[1]{\ensuremath{\mathbf{#1}}\xspace}
\newcommand{\x}{\boldvec{x}}
\newcommand{\w}{\boldvec{w}}


\author{Your name}
\date{Today}


\begin{document}

\maketitle


\begin{homeworkProblem}
\begin{enumerate}[a)]
    \item \solution{
    Your solution to sub-problem 1
    }
    \item Remaining sub-problems
    
\end{enumerate}
\end{homeworkProblem}

\begin{homeworkProblem}
The remaining problems.

Below are some examples for common equation formatting scenarios. Note the use of helper commands---you can also introduce new ones for your convenience.
\begin{itemize}
    \item Partial derivatives: $\pderiv{f}{x}$
    \item Vectors: $\x = \begin{bmatrix}
        1 \\ 2 \\ 3
    \end{bmatrix}$
    \item Matrices: $A = \begin{bmatrix}
        1 & 2 & 3 \\
        4 & 5 & 6
    \end{bmatrix}$
    \item Aligning equations with multiple lines: 
        \begin{align*}
            f(x) & = ax + b \\
                 & = cx + d
        \end{align*}
    \item Centered equations:
        \begin{gather*}
            a = 1, \quad b = 2 \\
            c = 3
        \end{gather*}
    \item Case statements:
        \begin{equation*}
            y = \begin{cases}
                1 & \text{if $x>0$} \\
                0 & \text{otherwise}
            \end{cases}
        \end{equation*}
\end{itemize}
\end{homeworkProblem}



\end{document}

